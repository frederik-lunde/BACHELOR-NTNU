\section{MVP user test feedback}
\label{app:mvp_user_test}

\subsubsection{\textit{\textbf{User test: participant 1}}}

The participant quickly found the “Add Website” function and submitted the form without issues, but expected an immediate confirmation or to see the new site appear in the list right away. The dashboard was described as visually appealing and rich in information, though the amount of data required some extra time to get oriented.  It was unclear what the main graph represented; whether the values were absolute or relative, and whether a tall bar indicated good or poor performance, especially since no clear time axis was shown. 

There was a wish to click directly on a point in the graph and jump to the corresponding incident in the “Website Details” view. A historical downtime graph that mirrored the overview was also requested. While the overall design language was well received, the status banner and search bar were seen as taking up too much vertical space. It was suggested that the dropdown sorter be replaced with visible buttons. Additionally, there was a lack of details about incident and incident history , and it was recommended to show more than just up/down status, ideally including response time data. 


\subsubsection{\textit{\textbf{User test: participant 2}}}

The participant completed all the tasks without issues but pointed out several structural elements that could be improved. Locating “Add Website” in the side navigation felt unintuitive; it was expected to be a prominent action button directly on the dashboard. The downtime graph was considered useful, but visually underwhelming. Suggestions included adding a clear time axis, colour coding, and duration bars to better indicate severity. 

After using the status filters, it became apparent that they remained active when returning to the Home view. This led to a recommendation to either reset them automatically or make the active filter more visible. When opening an alert, no cause was shown, so a short label like “Timeout” or “DNS Failure” was requested to make the alert content more informative.

The focus-mode icon was perceived as unclear, with a suggestion to replace it with a more familiar full-screen symbol. Additional ideas included adding a small “last 24h” graph to the dashboard and a thin downtime sparkline on each site card to support faster scanning.


\subsubsection{\textit{\textbf{User test: participant 3}}}

The third participant found the interface mostly intuitive but noted several missing visual cues. When attempting to return from Website Details to the dashboard, a dedicated Back button or breadcrumb was expected. Relying on the browser’s back arrow felt less convenient. Similar to other testers, the focus-mode icon caused confusion and a four-corner outline was suggested as a clearer way to signal screen enlargement.

The main dashboard graph lacked both a title and legend, which made its purpose unclear. Axis labels and a short explanatory description were requested to clarify the metric shown. The participant also couldn’t easily see how the graph related to the individual site cards and recommended using consistent colours or line styles to establish a visual connection.

While the icon style overall was well received, the border around the pin indicator stood out as distracting. A filled star without an outline was suggested as a cleaner alternative.