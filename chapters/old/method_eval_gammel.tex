\section{Evaluation of Methodology}
\label{sec:evaluation_of_ethodology}

The methodologies used in this project effectively supported an iterative, user-centered development process. By combining agile principles with structured requirement elicitation, user testing, and modern web technologies, the team was able to remain flexible to stakeholder input and align with the project's goals (appendix visjonsdokument?).

The agile, incremental development model enabled early delivery of a prototype and iterative improvements of the product based on real user feedback. The use of Kanban provided transparency and helped the team monitor progress and identify bottlenecks throughout the development process.

Requirement elicitation was strengthened through the use of multiple sources, including the assignment brief, stakeholder meetings, and feedback from user testing. Requirements were prioritized using a clear set of criteria that considered stakeholder needs, system criticality, and techical feasibiliy

User testing was a key strength of the project. Two structured tests, one of the prototype and one of the \acrshort{mvp}, provided insights at different stages of development. The use of the think-aloud protocol, semi-structured interviews, and the \acrshort{sus} questionnaire offered a valuable mix of qualitative and quantitative data for usability evaluation. 

Technology choices such as React, TypeScript, Node.js, and MySQL provided a modern and stable foundation for the development. The integration of automated tests, GitHub Actions, and end-to-end testing increased code reliability and helped maintain system stability. 